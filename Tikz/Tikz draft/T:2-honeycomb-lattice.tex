\documentclass{standalone}  %
\usepackage[margin=2cm]{geometry}
\usepackage{tikz,verbatim}
\usepackage[active,tightpage]{preview}
\PreviewEnvironment{tikzpicture}
\setlength\PreviewBorder{1pt}
%%%
\usetikzlibrary{arrows} 
\begin{document}
% Define the layers to draw the diagram
\pgfdeclarelayer{background}
\pgfdeclarelayer{foreground}
\pgfsetlayers{background,main,foreground}


\def \h { 0.57735026918}  % 1/sqrt(3) comprimento  y  de uma celula                                     even (White, 2y) to odd (black ,2y+1) 
\def \d { 0.28867513459}  % 1/ 2 sqrt(3)  distancia  y  entre celulas                                   odd (black , 2y-1) to even (white , 2y)
\def \c { 0.5}           % 1/2 distancia  x  entre celulas  

\def \v { 0.86602540378 }  % sqrt(3)/ 2   distancia  y  de duas celulas                                   odd-odd or even-even 

% We will also add a margin for the arrows
\def \mar {0.5}

\begin{tikzpicture}[>=latex]


\begin{comment}

    % For the arrows at the Black/Odd sites
        \foreach \y in {-1,0,1} {
           \node at (-1-2- \mar  , \v*\y ) (input1\y) { };
           \node at (1.5+2+ 1.6*\mar , \v*\y ) (out1\y) {};   %{ $\,$  $(2*\y+1)$ } ;
            \draw[->][line width=0.312mm, gray ] (input1\y) -- (out1\y);
        }
        
    % For the arrows at the White/Even sites
        \foreach \y in {-1,0,1} {
           \node at (-1 -2- \mar  , \v*\y -\d ) (input2\y) { };
           \node at (1.5 +2+ 1.6*\mar , \v*\y - \d ) (out2\y) {};   %{ $\,$  $(2*\y)$ } ;
            \draw[->][line width=0.312mm, gray ] (input2\y) -- (out2\y);
        }
    
    
    % For the Vertical arrows
        \foreach \x in {-6 , -4 , -2 , 0 , 2 , 4, 6 } {
            \node at ( \x*\c , -\mar  - 2*\d - \h  ) (input3\x) { };
            \node at ( \x*\c, 1.6*\mar + \d + \h ) (out3\x) { } ;
            \draw[->][line width=0.312mm, gray ] (input3\x) -- (out3\x);
        }

\end{comment}


% And now lets draw the circles

% For bigger latices I would first use a For in the \x direction in a unit cell, and then in the \y. Unfortunately the in two case I would find a circle that do not belong to a hexagon .... It easier to just write explicitly



\foreach \b in {0,1}{    
% For lines -1 and 0    

\foreach \a in {-2,0,2} {
	\node[circle, fill=black, draw=black!500, line width=0.45mm, inner sep=2.5pt, minimum size=4pt] (B21\b\a) at ( \a + \c, \v + 2*\v*\b ) {};
    \node[circle, fill=black, draw=black!500, line width=0.45mm, inner sep=2.5pt, minimum size=4pt] (B11\b\a) at ( -1+\a+\c, \v + 2*\v*\b  ) {};
    \node[circle, fill=white, draw=black!500, line width=0.45mm, inner sep=2.5pt, minimum size=4pt] (W21\b\a) at ( \a , -\d +\v+ 2*\v*\b  ) {};
    \node[circle, fill=white, draw=black!500, line width=0.45mm, inner sep=2.5pt, minimum size=4pt] (W11\b\a) at ( -1+\a , -\d + \v+ 2*\v*\b  ) {};
    \node[circle, fill=white, draw=black!500, line width=0.45mm, inner sep=2.5pt, minimum size=4pt] (W31\b\a) at ( 1+\a , -\d +\v + 2*\v*\b  ) {};
    \draw[draw=black!500, line width=0.4mm] (W11\b\a)--(B11\b\a)--(W21\b\a)--(B21\b\a)--(W31\b\a);
}   

% For lines 1 and 2

\foreach \a in {-2,0,2}{
	\node[circle, fill=black, draw=black!500, line width=0.45mm, inner sep=2.5pt, minimum size=4pt] (B22\b\a) at ( \a ,0 + 2*\v*\b ) {};
    \node[circle, fill=black, draw=black!500, line width=0.45mm, inner sep=2.5pt, minimum size=4pt] (B12\b\a) at ( -1+\a,0 + 2*\v*\b ) {};
    \node[circle, fill=black, draw=black!500, line width=0.45mm, inner sep=2.5pt, minimum size=4pt] (B32\b\a) at ( 1+\a,0 + 2*\v*\b ) {};
    \node[circle, fill=white, draw=black!500, line width=0.45mm, inner sep=2.5pt, minimum size=4pt] (W22\b\a) at ( \a + \c, -\d + 2*\v*\b ) {};
    \node[circle, fill=white, draw=black!500, line width=0.45mm, inner sep=2.5pt, minimum size=4pt] (W12\b\a) at ( -1 +\a + \c , -\d+ 2*\v*\b  ) {};
    \node[circle, fill=white, draw=black!500, line width=0.45mm, inner sep=2.5pt, minimum size=4pt] (W32\b\a) at ( 1 +\a  +\c, -\d+ 2*\v*\b  ) {};
   \draw[draw=black!500, line width=0.4mm] (B12\b\a)--(W12\b\a)--(B22\b\a)--(W22\b\a)--(B32\b\a)--(W32\b\a);
    \draw[draw=black!500, line width=0.4mm] (B12\b\a)--(W11\b\a) (B22\b\a)--(W21\b\a);
    \draw[draw=black!500, line width=0.4mm] (B32\b\a)--(W31\b\a);
}    

% For lines 3 and 4

\foreach \a in {-2,0,2}{
	\node[circle, fill=black, draw=black!500, line width=0.45mm, inner sep=2.5pt, minimum size=4pt] (B23\b\a) at ( \c+\a, -\v + 2*\v*\b  ) {};
    \node[circle, fill=black, draw=black!500, line width=0.45mm, inner sep=2.5pt, minimum size=4pt] (B13\b\a) at ( -1+\c+\a, -\v+ 2*\v*\b  ) {};
    \node[circle, fill=black, draw=black!500, line width=0.45mm, inner sep=2.5pt, minimum size=4pt] (B33\b\a) at ( 1 +\c+\a, -\v+ 2*\v*\b  ) {};
    \node[circle, fill=white, draw=black!500, line width=0.45mm, inner sep=2.5pt, minimum size=4pt] (W13\b\a) at ( \a, -\d -\v + 2*\v*\b ) {};
    \node[circle, fill=white, draw=black!500, line width=0.45mm, inner sep=2.5pt, minimum size=4pt] (W23\b\a) at ( 1+\a, -\d -\v + 2*\v*\b ) {};
    \draw[draw=black!500, line width=0.4mm] (B13\b\a)--(W13\b\a)--(B23\b\a)--(W23\b\a)--(B33\b\a);
    \draw[draw=black!500, line width=0.4mm] (B13\b\a)--(W12\b\a) (B23\b\a)--(W22\b\a);
    \draw[draw=black!500, line width=0.4mm] (B33\b\a)--(W32\b\a);
}    
}

\draw[->,draw=blue!90!black!60!, line width=0.4mm] (0.01,-\d-\v - 0.01 + \v + \d)--( -1+\c - 0.08, -\d +0.125 + \v +\d ) node[left] { \textcolor{blue!90!black!80!}{$n_{2}\, $} };
\draw[->,draw=blue!90!black!60!, line width=0.4mm] ( -0.01,-\d -\v - 0.01 + \v + \d)--(\c + 0.08, -\d + 0.125 + \d+ \v ) node[right] {\textcolor{blue!90!black!80!}{$\, n_{1}$}};

\draw[->,draw=red!70!black!70!, line width=0.4mm] (-2,0)--( -2+\c+0.085, -\d-0.05 ) node[midway,below] { \textcolor{red!70!black!70!}{$e_{2} \, $} };
\draw[->,draw=red!70!black!70!, line width=0.4mm] (-2,0)--( -3+\c -0.085, -\d- 0.05 ) node[midway,below] { \textcolor{red!70!black!70!}{$\,e_{3}$} };
\draw[->,draw=red!70!black!70!, line width=0.4mm] (-2,0)--( -2, -\d+\v+0.11 ) node[midway,right] { \textcolor{red!70!black!70!}{$e_{1}$} };

\end{tikzpicture}
\end{document}