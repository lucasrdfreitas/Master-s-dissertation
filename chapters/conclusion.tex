%The result of this work reconcile the interpretation of recent experimental measure in \acrshort{rucl} with the idea of a \acrlong{kqsl}. As shown, it is possible to explain a gapless behaviour in the gapped phase for Kitaev materials by means of Majorana fermions localised in defects. 
%three paragraphs
% 1st:   restate what the work was about
%in this thesis a calculate the 
% 2nd:    summarise the main points, clarify, concern
% bases in  ... i was able to study ... and find that ... 
% 3rd:    draw to a close;   final observation, consequences 
% future research ...  


This work started with the intent of explaining \acrshort{nmr} results for the Kitaev material \acrshort{rucl}. 

There are still many unanswered questions in the study of \acrshort{kqsl}s. They span the whole spectrum from its experimental realization to how to implement fault-tolerant topological quantum computation. In this work, I have cleared one unexpected experimental measurement for the Kitaev material \acrshort{rucl} in light of the theoretical Kitaev model.

In this thesis,  I have studied various properties of the Kitaev model, from its microscopic description to an effective theory for its Majorana edge modes% in the mean-field approximation
. I used these techniques to analyze the model in the B phase with an applied magnetic field.  In this phase, the system is in a gapped phase, and the low-energy excitations are the Majorana fermions localized in the edge.  To explain the presence of low-energy excitation in the bulk, I propose a model that describes how the Majorana fermions can be localized into linear defects inside the material. In Kitaev materials candidates, these defects are expected to occur, for instance along partial dislocations associated with stacking faults.

First, I described the pure isotropic Kitaev model in a magnetic field and analyzed the gapless Majorana edge modes. These gapless modes are the main element for the low-energy theory. To study how these modes behave in a linear defect I investigated a system in which the layer is coiled into a quasi-cylindrical geometry system in which the two edges are close to each other (see figure \ref{fig:4-cylinder}).  This is a model for a linear defect. The distance between the edges of the defect controls the interaction. 

Secondly, I analyzed the interaction between the two edges of the defects through the mean-field approximation.  This analysis reveals a phase transition as a function of the exchange coupling across the defect. If the defect is too narrow, the edges are close together, and the material is in a uniform phase with all Majorana fermions in the bulk with a non-zero energy gap. However, if the defect is reasonably wide the system is then in a phase with localized Majorana fermions without a gap in the defect edge. This investigation sustains my original hypothesis.

Thirdly, I focused on the phase with gapless Majorana fermions at the edge and studied the theory for these low-energy excitations in the continuum. I was able to describe the dynamics in an effective model for the Majorana fermions. All the parameters of this theory are determined in terms of the microscopic parameters. I used this theory to calculate the spin-lattice relaxation rate. The contribution $1/T_1 \sim T^3$ for the modes localized in the defects is in agreement with the experimental result that this work was perusing to understand.

 
The study proposes a different% consequence of this present study proposes a different
 interpretation of the experimental result of Ref. \cite{Zheng-gapless2017}. Materials with diverse concentrations of linear defects provide a way to validate our work. Strains in the material or other form to favor partial dislocation defects possibly produce linear defects.

%Importantly, this work provides the awareness that some experimental results in Kitaev material candidates can be understood using the current theory for the Kitaev model. 
Importantly, this work indicates that puzzling experimental results may be reconciled with the current theory for \acrshort{kqsl}, supporting the interpretation that Majorana fermions may indeed be present in the intermediate-field phase of \acrshort{rucl}.
