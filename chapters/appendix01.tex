
\textcolor{red!40!black}{
In this appendix I want to derive the relation between the correlation function and the imaginary part of the susceptibility [e.g. chapter 09 of Piers Coleman many body physics, \cite{coleman2015}, and chapter 07 of Altland and Simons, \cite{altland2010condensed}]. This chapter will be divided in three section:
\begin{itemize}
    \item (Quantum) Fluctuation-dissipation theorem in real time;
    \item (Quantum) Fluctuation-dissipation theorem in imaginary time;
    \item Spin spectroscopy and NMR .
\end{itemize}}


% NMR = Nuclear magnetic resonance ; it is the use of nuclear magnetic absorption lines to probe local spin environment in a material. The nuclear spin interact with the applied magnetic field due to the Zeeman interaction, this processes gives rise to a resonant absorption line in the microwave domain.  The interaction of the nucleus' spin with the surrounding spin and orbial moments produces a \textit{Knight shift} of the absorption line (also broadens the line, the width is associated with the nuclear spin relaxation rate).
% Knight shift : shift in the magnetic resonance line, it is directly related with the expectation value of the hyperfine field $B_{\text{hf}}$
% it is known as magnetic resonance imaging (MRI) for medical usage

