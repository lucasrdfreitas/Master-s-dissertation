





The interplay of topology and strongly correlated phenomena gives rise to fascinating physics such as the fractionalization of elementary particles.  An important example is the fractional quantum Hall effect, in which the electronic degrees of freedom split into quasiparticles with fractional electron charge. Another prominent example is the central theme of this work: the \acrfull{qsl} \cite{Balents2010, Savary_2016, Knolle_2019, Misguich2011}. In magnetic insulators, these phases of matter exhibit long-range coherence, but fluctuation prevents the formation of magnetic order even at zero temperature.

A remarkable example of \acrshort{qsl} is the \acrfull{kqsl} proposed as a toy model by A. Kitaev in 2006 \cite{Kitaev_2006}. This strongly interacting spin system has as its elementary excitation $\mathbb{Z}_2$-vortices and Majorana fermions. Upon applying a magnetic field, the Majoranas become gapped with a Chern insulator spectrum. The non-trivial topology of this insulator gives rise to chiral Majorana modes at the edge. Unlike the fractional quantum Hall effect, the edge modes with opposite chiralities in \acrshort{kqsl} are electrically neutral. %occur in two fermionic species of opposite chiralities and are electrically neutral. %and Chern number  $\pm 1$. 

In previous work, \cite{Kitaev_2003}, Kitaev had proposed that non-Abelian anyons could be used to realize a physical error-correction quantum code. The anyonic system provides a realization of a quantum memory that is protected from decoherence. This remarkable property raised attention for theoretical models with this type of particle since the current major limitation to construct a useful quantum computer is the frequent errors caused by decoherence. The quantum computers (theoretically) made using these models are named topological quantum computers \cite{Nayak2008, Stern1179}.  The Kitaev model was constructed with the purpose of creating a two-dimensional spin model whose elementary were non-Abelian anyons.

The Kitaev model was not created with a physical realization in mind. Nevertheless, a few years later Jackeli and Khaliullin \cite{Jackeli_2009} proposed a mechanism for a Mott insulator with dominant Kitaev interaction. The first studied materials were the iridates, $\text{A}_2\text{IrO}_3$ (A = Na,Li), which realize the Kitaev model plus additional small non-Kitaev interaction. The model that include those non-Kitaev interactions is called \textit{extended Kitaev model} or $J\text{-}K\text{-}\Gamma$, $J$ for the Heisenberg interaction, $K$ for the Kitaev interaction, and $\Gamma$ for the off-diagonal interactions.  % isotropic exchange (J) and anisotropic  The material \acrshort{rucl} became more popular as was discovered that 

Even though the non-Kitaev interactions are small in the iridate materials, they are sufficient to drive a zigzag order for weak magnetic fields \cite{Choi-2012,Ye-2012}. Therefore, the iridates do not  have a \acrshort{qsl} ground state. Furthermore, for strong fields, the material goes through a phase transition to a paramagnetic phase, a trivial partially polarized state. %(forced ferromagnetic order). 
There is a small window of magnetic fields in which a \acrshort{qsl} may exist. %A similar, but slightly better, situation happens for \acrfull{rucl}. The similar limitation happens in the magnetic field domain, however the Néel temperature ($T_N$) is much smaller for this ruthenium material than for iridate ones. % This behaviour is probably due d
A similar situation happens for \acrfull{rucl}. However, it is slightly better since the Néel temperature ($T_N$) is much smaller for this ruthenium material than for iridate ones.

%%%%%%%%%%%%%%%%%%%%%%%%%%%%%%%%%%
%%%%%%%%%%%%%%%% Rewrite this paragraph
One experimental group tested the spin-orbit Mott insulator \acrshort{rucl} and found evidence for a thermal quantum Hall effect \cite{Kasahara_2018,yokoi2020halfinteger}. % Unlike the Integer or Fractional Hall Effects, the Kitaev Spin Liquid excitation are neutral. 
While the elementary excitation of \acrshort{kqsl} are neutral, they can carry energy and therefore contribute to  heat transport. The experiment was done using a temperature gradient %introduce a current of phonons in the edge to interact thermally with the Majorana. 
The theory for \acrshort{kqsl} predicts that the \acrshort{rucl} exhibits a  quantization of the transverse thermal conductivity $\kappa_{xy}$. The experimental results are not sharp and clear, but the authors of \cite{Kasahara_2018} believe they found the thermal Hall effect in a region of moderate magnetic fields 7--9T.
% see the articles  \cite{Ye_2018,Aviv_2018}
The value measured for $\kappa_{xy}$ is not equal to the ideal conductivity of an isolated edge mode. There are proposals to explain this difference because of the coupling between the phonons and the Majorana edge modes \cite{Ye_2018,Aviv_2018}.



%\textcolor{red!40!black}{One candidate material for the \acrfull{kqsl} is \acrshort{rucl}. It is believed to behave as a \textbf{gapped} \acrshort{qsl} in a region of intermediate field $(8\sim10\text{ Tesla}$). In this introduction I want to give a short review about the recent experiments that rises the doubts about the \acrshort{rucl} to be a gapped or a gapless phase inthis region.\begin{itemize}    \item Introduction to Kitaev material (which \textbf{articles ?}   );    \begin{itemize}        \item Articles about \acrlong{qsl} : ;        \item Quantum computation (application) : \cite{Nayak2008};        \item About the theory of the \acrlong{kqsl}: \cite{Kitaev_2006, Jackeli_2009}    \end{itemize}    \item Review of experiments about the gap in  \acrshort{rucl} ( e.g. \cite{Zheng-gapless2017, Baek2017, Nagai_2020, tanaka2020} );    \item Interaction between chiral Majorana edges modes : \cite{Aasen_2020} .\end{itemize}}

%A fractional excitation is non-local in the spin operator. In fact, for a spin-half system the ladder operator for spin $S_{i}^{+}$ and $S_{i}^{-}$ can change the total magnetisation $S^{z}_{\text{total}}$ in integer values (units of $\hbar = 1$).


A recent \acrfull{nmr} experiment found unexpected results \cite{Zheng-gapless2017}. From the theory of Kitaev materials,  \acrshort{rucl} should behave as a gapped \acrshort{qsl} under not too strong magnetic fields. This behavior was experimentally verified in  thermal Hall measurements in \cite{Kasahara_2018,yokoi2020halfinteger} and specific heat measurements \cite{tanaka2020}. However,  this \acrshort{nmr} experiment found a spin-lattice relaxation rate of $T_1^{-1} \propto T^3$, which is rather expected for a gapless fermionic system. In the words of the authors of \cite{Zheng-gapless2017}: \say{\textit{our NMR results show unequivocally that the spin excitations themselves are gapless, and thus such proximate Kitaev physics appears to be excluded. [...] This result cannot be reconciled with the behavior of conventional quantum magnets or of the pure or generic Kitaev QSL}}.  

Another \acrshort{nmr} experiment \cite{ Baek2017} claimed to observe a behavior $T_1^{-1} \propto e^{-\Delta/T}$, but only in the regime of large magnetic fields where the gap must be associated with magnons in the trivial paramagnetic phase. In \cite{Nagai_2020}, the temperature dependence of $1/T_1$ was fitted to a double exponential in an attempt to extract two different energy scales associated with Majorana fermions and vortices.
%On the other hand, was experimentally found others $T_1$ behaviours, e.g. $T_1^{-1} \propto e^{-\Delta/T}$ \cite{ Baek2017} or more complicated \cite{Nagai_2020}, that agree with the gapped nature of the excitation in \acrshort{rucl}.
A very recent thermal transport experiment,\cite{Hentrich_2020}, founded evidence for defect-induced low-energy excitations in the longitudinal heat conductivity for \acrshort{rucl}.

On theoretical grounds, Kitaev materials under moderate magnetic fields should exhibit gapless Majorana fermions at the edge, but the \acrshort{nmr} experiments on an ideal material would probe the bulk excitation which is gapped. In this work, I explore the possibility that the observed gapless mode in the bulk is due to Majorana fermions localized along with linear defects in the material. % A reasonable explanation to found gapless excitation in the bulk is due to the presence of localised Majorana fermions in linear defects of the material. 
It is known that stacking faults in layered materials may induce the formation of linear defects such as partial dislocations \cite{solyom-solids}.
Indeed, \acrshort{rucl} is a quasi-two-dimensional material with weakly interacting layers that can easily slip on one another producing stacking faults \cite{Kim_Kee_2016,Cao_2016,Ran2017,Zheng-SM2017}.

This work is a study of linear defects in the pure Kitaev model. As a first theoretical model, I focus on a particular type of linear defects in which the structure resembles a zigzag edge. The toy model for this defect consists of seaming two semi-infinite \acrshort{kqsl}. At my starting point, both subsystems contain chiral edge modes, which I then couple by an exchange interaction that is weaker along with the defect than in the bulk. My model is inspired by Ref.\cite{Aasen_2020}, which noted that there is a critical value of the exchange interaction between edges below which the chiral modes remain decoupled. From the point of view of the  low-energy  effective theory, the reason is that the weak interaction between emergent Majorana fermions is irrelevant in the renormalization group sense. %The interaction between the edges is irrelevant, and for weak interaction the model is a effective a free Majorana system. Some of the idea of proximate two Kitaev materials was inspired by \cite{Aasen_2020}, which uses this mechanism for a different purpose.

In contrast to the field theoretical arguments of Ref.\cite{Aasen_2020}, here I first study the critical condition for seaming the one-dimensional Majorana modes directly in the lattice model. First, I calculate the spectrum using a self-consistent mean-field approximation for the exchange interaction along with the defect. Having identified the regime of decoupled gapless modes, I proceed by explicitly deriving the representation of the spin operator in the low-energy sector as a Majorana fermion bilinear. Following, I then use the effective field theory to compute the dynamical spin-spin correlation function which determines the spin-lattice relaxation rate at low temperatures. I find the power-law dependence $T_1^{-1} \propto T^3$, in agreement with the experimental results of Ref.  \cite{Zheng-gapless2017}.

This dissertation is organized as follows. In chapter \ref{ch:2}, I will set the notation and review the important aspects of the Kitaev model. Complementary, the description of the model in a more convenient geometry with zigzag edges is presented in chapter \ref{ch:3}. The defects are described %The description of the defects is made
in chapter \ref{ch:4} in which mean-field approximation  is used to deal with the interaction across the \say{edges} of the defect. The approximation gives ground to describe the low-energetic excitation localized in the defect as Majorana fermions, the passage from the microscopic theory to this effective theory is described in chapter \ref{ch:5}. The spin-lattice relaxation rate, $1/T_1$, is calculated in chapter \ref{ch:6}.















